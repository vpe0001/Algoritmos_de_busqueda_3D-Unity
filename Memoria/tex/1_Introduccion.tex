\capitulo{1}{Introducción}

El vehículo autónomo es uno de los avances tecnológicos que más impacto va a tener en la sociedad en los próximos años. Tal es así, que las mayores compañías ya están trabajando en el cambio que se avecina.

Brian Krzanich, director de Intel declaró \cite{webtelegraphselfdrivingcars} que el cambió será tan explosivo que las compañías que no se preparen se enfrentarán al fracaso y a la extinción. Se esperá que las cantidades de dinero que mueva este nuevo sector sean gigantescas, 800 mil millones de dolares para el año 2035 y hasta 7 billones de dolares en lo siguientes 15 años \cite{webtelegraphselfdrivingcars}.

Este nuevo sector no solo involucra a la industria automovilística y sus empresas tradicionales. Empresas del sector de la informática, como Intel, Nvidia, Google, Apple están invirtiendo en el desarrollo de las tecnologías necesarias. Y muchas empresas mas pequeñas que trabajan en tecnologías necesarias para el coche autónomo han sido compradas por gigantes tecnológicos, como Mobileye comprada por Intel especializada en visión computarizada, Primesense comprada por Apple especializada en los sensores de tres dimensiones, o Waze comprada por Google especializada en los mapas para la navegación GPS \cite{webtechcrunchselfdrivingcars}.

Pero no solo es una revolución en la economía. Supondrá también un cambio social. Los vehículos autónomos tienen menos accidentes que los humanos \cite{webfastcompanyselfdrivingcars}, son más ecológicos el reducir las emisiones de $CO_2$ \cite{webbusinessinsiderdrivingcars}, y cambiarán la manera de la sociedad de desplazarse. Los coches autónomos no solo permiten usar el tiempo que las personas pasan al volante en otras actividades, también permiten más independencia en sus desplazamientos a personas mayores o con discapacidad. Cambia la forma en que la gente puede compartir vehículos y desplazamientos como lo ha hecho Uber. El espacio de las ciudades destinado aparcamiento podría cambiar drásticamente con vehículos que se aparcan solos. También se verá una reducción del tráfico y la congestión en las carreteras debido a su conducción más eficiente.

Como hemos mencionado, el número de tecnologías necesarias para construir un coche autónomo es enorme. Desde el propio vehículo, todos los sistemas para la detección del entorno y la visión computarizada, hasta todo el software necesario que hay que desarrollar. Algoritmos que permitan la percepción, conocer el entorno y las variables imprevistas que surjan. Algoritmos que permiten conocer la localización exacta en tiempo real. Algoritmos que permitan planificar las acciones que debe llevar a acabo el vehículo. Y algoritmos que permitan la locomoción del vehículo y desplazarlo siguiendo la planificación.

De todas estás tecnologías, en este proyecto nos hemos fijado en las dos últimas. Al usar un entorno tridimensional para realizar la simulación, podemos conocer el entorno en el que se encuentra el vehículo así como su localización. De esta forma podemos centrarnos en el desarrollo de algoritmos permitan planificar una ruta y que le indiquen a un vehículo como desplazarse de forma autónoma en un entorno continuo en un espacio tridimensional, como ocurre en la realidad. También nos permite ver las nuevas dificultades que plantea y adaptar los algoritmos que hemos ido viendo en la carrera en un entorno más realista, así como otros nuevos que permiten entender y solventar los problemas a los que se enfrentan los coches autónomos que circulan por las calles.
