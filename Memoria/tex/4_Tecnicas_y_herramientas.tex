\capitulo{4}{Técnicas y herramientas}

\section{Programas usados}

\subsubsection{Unity}
\textit{Unity}\footnote{Unity 3D \url{https://unity3d.com/}} un motor para juegos usado principalmente para desarrollar videojuegos y simulaciones. Que sea un motor para juegos quiere decir que está formado por varios motores a su vez, como el motor gráfico y el motor de físicas.

\textit{Unity} proporciona un editor donde se puede crear escenas junto con un gran número de herramientas, ejemplos y modelos para crear escenarios tanto 2D y principalmente 3D.

Tiene una licencia de uso «por pasos» de ingresos obtenidos por el contenido creado con él. Existen distintas versiones de la licencia según el uso que se vaya hacer, así como de la cantidad de ingresos que se generen con su uso. En la versión Personal, se puede usar libremente siempre que los ingresos generados por el contenido usado no supere los 100 mil dólares anuales, y es la versión de la licencia usada. Las siguientes versiones de la licencia requieren un pago de una cuota anual por cada usuario y el límite de ingresos por el uso comercial del contenido creado va aumentando.

Para el proyecto probamos de forma descendiente las distintas versiones de \textit{Unity} para Linux en un xUbuntu 16.04, debido a que las últimas versiones producían errores que cerraban el editor al iniciarlo o al poco tiempo después de abrirse. Creemos que esto se debía a un problema de compatibilidad entre los \textit{drivers} de la tarjeta gráfica que no tenían soporte para algunas de las nuevas características de \textit{Unity}. La primera versión estable que funcionaba en el equipo de desarrollo fue la versión 5.3.6f1, y ha sido la usada para su realización.

\subsubsection{Texmaker}
Para la realización de la memoria se ha usado latex con el editor Texmaker\footnote{Texmaker \url{http://www.xm1math.net/texmaker/}}.
Texmaker es un editor libre con licencia GPL.

Texmaker incluye varias herramientas como el corrector ortográfico o el visor de pdf que facilitan la creación de los documentos.

La versión usada ha sido la 4.4.1.

\subsubsection{SmartGit}
SmartGit\footnote{SmartGit \url{http://www.syntevo.com/smartgit/}} es un gestor gráfico para git que soporta github y que facilita la realización de \textit{commits}, \textit{push} y \textit{pull} de los repositorios del proyecto.

Tiene una licencia no comercial que permite el uso de forma gratuita a desarrolladores de código abierto, profesores, estudiantes, desarrollos no comerciales y también organizaciones sin ánimo de lucro.

La versión que hemos usado ha sido la 17.0.3

\subsubsection{GitHub y ZenHub}
La plataforma elegida para los repositorios ha sido GitHub, principalmente por su compatibilidad con Zenhub\footnote{Zenhub \url{https://www.zenhub.com/}} que es la herramienta de gestión de proyectos que hemos usado.

Zenhub es una herramienta que permite la planificación y control del proyecto. Es un complemento de Firefox que se integra con github y añade funciones como las \textit{boards} para manejar el control de las \textit{issues} y las \textit{milestones} del proyecto, y que permite ver los gráficos del progreso y \textit{burnouts} que se ha realizado.

La versión utilizada ha sido la 2.34.2

\subsubsection{Remarkable}
Remarkable\footnote{Remarkable \url{https://remarkableapp.github.io/}} es un editor del lenguaje \textit{markdown} usado para crear archivos de texto plano que puedan ser convertidos a HTML, PDF o otros formatos.

Lo hemos usado para crear y editar el archivo readme.md del repositorio de GitHub.

El programa tiene una licencia libre de código abierto que permite usarlo sin limitaciones, incluido fines comerciales.

La versión utilizada ha sido la 1.87.
