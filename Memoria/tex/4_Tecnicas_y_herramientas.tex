\capitulo{4}{Técnicas y herramientas}

Esta parte de la memoria tiene como objetivo presentar las técnicas metodológicas y las herramientas de desarrollo que se han utilizado para llevar a cabo el proyecto. Si se han estudiado diferentes alternativas de metodologías, herramientas, bibliotecas se puede hacer un resumen de los aspectos más destacados de cada alternativa, incluyendo comparativas entre las distintas opciones y una justificación de las elecciones realizadas. 
No se pretende que este apartado se convierta en un capítulo de un libro dedicado a cada una de las alternativas, sino comentar los aspectos más destacados de cada opción, con un repaso somero a los fundamentos esenciales y referencias bibliográficas para que el lector pueda ampliar su conocimiento sobre el tema.


\section{Programas usados}

\subsubsection{Unity}
Unity un motor para juegos usado principalmente para desarrollar viejuegos y simulaciones. Que sea un motor para juegos quiere decior que está formado por varios motores a su vez, como el motor gráfico y el motor de físicas.

Unity proporciona un editor donde se puede crear escenas junto con un gran número de herramientas, ejemplos y modelos para crear escenarios tanto 2D y principalment 3D.

Tiene una licencia de uso por pasos de ingresos obtenidos por el contenido creado con el. En la versión Personal, su puede usar libremente siempre que los ingresos generados por el contenido usado no supere los 100 dólares anuales, y es la versión de la licencia usada.

Para el proyecto probamos de forma descente las distintas versiones de Unity para linux en un xUbuntu 16.04, debido a que las últimas versiones producían errores que cerraban el editor al iniciarlo o al poco tiempo despues de abrirse. La primera versión estable que funcionaba en el equipo de desarrollo fue la versión 5.3.6f1, y ha sido la usada para su realización.

\subsubsection{Texmaker}
Para la realización de la memoria se ha usado latex con el editor texmaker.
Texmaker es un editor libre con licencia GPL.

Texmaker incluye varias herramientas como el corrector ortográfico o el visor de pdf que facilitan la creación de los documentos.

La versión usada ha sido la 4.4.1.

\subsubsection{SmartGit}
SmartGit es un gestor gráfico para git que soporta github y que facilita la realización de commits, push y pull de los repositorios del proyecto.

Tiene una licencia no comercial que permite el uso de forma gratuita a desarrolladores de código abierto, profesores, estudiantes, desarrollos no comerciales y también organizaciones sin ánimo de lucro.

La versión que hemos usado ha sido la 17.0.3

\subsubsection{GitHub y ZenHub}
La plataforma elegida para los repositorios ha sido github, principalmente por su compatibilidad con Zenhub que es la herramienta de gestión de proyectos que hemos usado.

Zenhub es una herramienta que permite la planificación y control del proyecto. Es un complemento de firefox que se integra con github y añade funciones como las boards para manejar el control de las issues y las milestones del proyecto, y que permite ver los gráficos del progreso y burnouts que se ha realizado.

La versión utilizada ha sido la 2.34.2

